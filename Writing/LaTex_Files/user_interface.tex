\section{User Interface}
The user interface (UI) for the robot is designed to facilitate interactive control and real-time monitoring. It integrates command parsing, telemetry data handling, and communication protocols such as Inter-Integrated Circuit (I2C) and Universal Asynchronous Receiver-Transmitter (UART). Users can send commands to the robot and receive real-time telemetry feedback, ensuring efficient control and monitoring.

Communication is structured around well-defined command and telemetry packets, implemented in \texttt{comm.hpp}. These packets support multiple data formats, ensuring compatibility with various communication protocols. The functions responsible for sending and receiving telemetry data are listed in Tables~\ref{tab:uart_methods} and~\ref{tab:i2c_methods}.

\subsection{UART Communication}
The Universal Asynchronous Receiver-Transmitter (UART) protocol is used for serial communication, typically with a Bluetooth module or a computer for debugging and remote control. The communication is initialized at a baud rate of \texttt{9600}, which is commonly used for Bluetooth modules \cite{bluetooth}.

\begin{table}[h]
	\centering
	\caption{UART Communication Methods}
	\begin{tabular}{|l|l|l|}
		\hline
		\textbf{Packet Type} & \textbf{Function} & \textbf{Description} \\ \hline
		\multirow{4}{*}{Command Packets} & \texttt{readUartBytes()} & Reads command data as raw bytes. \\ \cline{2-3}
		& \texttt{readUartASCII()} & Reads command data as an ASCII string. \\ \cline{2-3}
		& \texttt{sendUartBytes()} & Sends command data as raw bytes. \\ \cline{2-3}
		& \texttt{sendUartASCII()} & Sends command data as an ASCII string. \\ \hline
		\multirow{4}{*}{Telemetry Packets} & \texttt{sendUartBytes()} & Sends telemetry data as raw bytes. \\ \cline{2-3}
		& \texttt{sendUartASCII()} & Sends telemetry data as an ASCII string. \\ \cline{2-3}
		& \texttt{readUartBytes()} & Reads telemetry data as raw bytes. \\ \cline{2-3}
		& \texttt{readUartASCII()} & Reads telemetry data as an ASCII string. \\ \hline
	\end{tabular}
	\label{tab:uart_methods}
\end{table}

\subsection{I2C Communication}
Inter-Integrated Circuit (I2C) communication is used for short-range, two-wire serial communication. The robot operates as an I2C slave with a predefined address (SLAVE\_ADDR = 8), allowing it to receive commands and transmit telemetry data over the I2C bus.

\begin{table}[h]
	\centering
	\caption{I2C Communication Methods}
	\begin{tabular}{|l|l|l|}
		\hline
		\textbf{Packet Type} & \textbf{Function} & \textbf{Description} \\ \hline
		\multirow{4}{*}{Command Packets} & \texttt{sendI2CBytes(addr)} & Sends command data as raw bytes to the specified address. \\ \cline{2-3}
		& \texttt{sendI2CASCII(addr)} & Sends command data as ASCII to the specified address. \\ \cline{2-3}
		& \texttt{readI2CBytes(addr)} & Reads command data as raw bytes from the specified address. \\ \cline{2-3}
		& \texttt{readI2CASCII(addr)} & Reads command data as ASCII from the specified address. \\ \hline
		\multirow{4}{*}{Telemetry Packets} & \texttt{sendI2CBytes()} & Sends telemetry data as raw bytes. \\ \cline{2-3}
		& \texttt{sendI2CASCII()} & Sends telemetry data as ASCII. \\ \cline{2-3}
		& \texttt{readI2CBytes(addr)} & Reads telemetry data as raw bytes from the specified address. \\ \cline{2-3}
		& \texttt{readI2CASCII(addr)} & Reads telemetry data as ASCII from the specified address. \\ \hline
	\end{tabular}
	\label{tab:i2c_methods}
\end{table}

\subsection{Sending Commands}
Users can send commands to control the robot’s movement, rotation, or stop function. Commands follow a predefined structured format and can be transmitted via either I2C or UART. The available commands are listed in Table~\ref{tab:commands}.

\textbf{ASCII Format}:
The commands sent in ASCII is formatted as a comma-separated string:
\begin{lstlisting}[]
	<command>,<command_value>,<command_speed>
\end{lstlisting}

\begin{table}[H]
	\centering
	\caption{List of Commands and Corresponding Values}
	\label{tab:commands}
	\begin{tabular}{|c|c|l|}
		\hline
		\textbf{Command} & \textbf{Value} & \textbf{Description} \\ \hline
		\texttt{Stop}     & 0              & Stops the robot's movement. \\ \hline
		\texttt{Move}     & 1              & Moves forward or backward (in cm) at a given speed. \\ \hline
		\texttt{Rotate}   & 2              & Rotates the robot by a specified angle (in degrees) at a given speed. \\ \hline
		\texttt{INVALID}  & 3              & Represents an invalid or unrecognized command. \\ \hline
	\end{tabular}
\end{table}

\textbf{Example Command:}
\begin{itemize}
	\item Stop the robot: \texttt{0}
	\item Move forward 100 cm at 50\% speed:  \texttt{1,100,50}
	\item Rotate 90° at 30 speed\%: \texttt{2,90,30}
\end{itemize}

\subsection{Receiving Telemetry Data}
The robot continuously monitors its state and environment using onboard sensors. This data is packaged into a structured format and transmitted back to the user for real-time monitoring and feedback. Telemetry data includes:
\begin{itemize}
	\item \textbf{Yaw Angle}: The robot's current orientation in degrees.
 	\item \textbf{Distance Traveled}: The total distance traveled by the robot in centimeters.
	\item \textbf{Ultrasonic Distance}: The distance to the nearest obstacle, as measured by the ultrasonic sensor in centimeters.
\end{itemize}

\textbf{ASCII Format}:
The telemetry data recieved in ASCII is formatted as a comma-separated string:
\begin{lstlisting}[]
	<yaw_angle>,<distance_traveled>,<ultrasonic_distance>
\end{lstlisting}
\textbf{Example Output}:
\begin{lstlisting}[]
	45,200,30
\end{lstlisting}

This indicates a yaw angle of 45 degrees, a distance traveled of 200 cm, and an ultrasonic reading of 30 cm.
